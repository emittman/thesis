% \section*{DEFINITIONS USED IN HIERARCHICAL GLFP METHODOLOGY}

\section*{Truncation adjusted Kaplan-Meier estimate of lifetime}
\label{sec:trunc-adj}
We first start with a nonparametric estimate of the empirical cdf for each drive-model using the Kaplan-Meier estimator.  With left truncation, however, the standard Kaplan-Meier estimator for drive-model $g$, denoted by
$\widehat{F_g(t)}_{KM}$, is conditional on survival up to
$t_{g,\text{min}}^L$, the shortest reported running time of all units
of drive-model $g$ for which records are available. To produce
unconditional estimates, we adapt the adjustment method outlined by \citet{meeker} (Chapter 11).  For each drive-model we select
$t_{g,\text{min}}^L$, the smallest left truncated time in the sample.
By sampling from the full posterior distribution, since
$\Pr(T>t_{g,\text{min}}^L|\theta_g)$ (the probability a hard drive has
survived up to $t_{g,\text{min}}^L$) is a function of the model
parameters, we can easily compute its posterior median,
$\widehat{A}_{\text{med}} = \widehat{\Pr}(T>t_{g,\text{min}}^L|\theta_g)$. We compute the adjusted estimate by

$$\widehat{F(t)}_{KMadj} = \widehat{A}_{\text{med}} + \left(1 - \widehat{A}_{\text{med}}\right)\widehat{F_g(t)}_{KM},\; t>t_{g,\text{min}}^L.$$

While this adjustment is negligible for the majority of drive-models, five drive-models receive upward adjustments of greater than 5 percent and the estimated time to failure distribution of one drive-model (30) was adjusted by nearly 16 percent, in part because the shortest truncation time for all observed units was approx. 2.3 years.

\section*{Definition of global average for Model 4}
\label{global-avg}
In Section \ref{model-assessment}, to illustrate the concept of shrinkage in our hierarchical model, we refer to a ``global average" which represents an average GLFP cdf for the entire population, $\op{H}(\cdot|\bar{\pi},\mu_1,\sigma_1,\bar{\mu}_2,\bar{\sigma}_2)$. Since $\mu_1$ and $\sigma_1$ are shared across drive-models, these can already be interpreted as ``global". For the parameters that vary across drive models, we select values corresponding to the medians of the hierarchical distributions (\ref{eq:hier-model}) conditional on the hyperparameters. In particular, we set
$$\bar{\pi}=\op{logit}^{-1}(\eta_{\pi}),\;\bar{\mu}_2=\eta_{t_2} - m_{\sigma_2}\Phi^{-1}(.2) \mbox{ and } \bar{\sigma}_2= m_{\sigma_2}.$$
Let $J(\cdot|a,b),\;J^{-1}(\cdot|a,b)$ denote the cdf and inverse cdf, respectively, for a log-normal distribution with log-location parameter $a$ and log-scale parameter $b$. Then
$$m_{\sigma_2}=J^{-1}[0.5 \cdot J(1|\eta_{\sigma_2},\tau_{\sigma_2})|\eta_{\sigma_2}, \tau_{\sigma_2}],$$
which is the median of a log-normal distribution with parameters $\eta_{\sigma_2}$, and $\tau_{\sigma_2}$, truncated to the interval $(0, 1)$.

We use posterior draws, $H(\tilde{t}|\eta_\pi^{(s)}, \mu_1^{(s)}, \sigma_1^{(s)}, \eta_{t_2}^{(s)},\eta_{\sigma_2}^{(s)},\tau_{\sigma_2}^{(s)})$, $s=1,2,\ldots,S$ to estimate the global average pointwise, . This computation is similar to that shown in (\ref{pointwise-medians}).