Advances in modern computing have encouraged statisticians to fit larger and larger models to larger and more complex data sets. Bayesian hierarchical models are a class of models, suitable for a wide range of applications, that offer the analyst flexibility and for which general strategies for inference have been developed. In this work we present two such models, both motivated by real applications, and develop methodologies for performing inference with them.

First, we present a Bayesian nonparametric hierarchical regression model for gene expression profiling data. In gene profiling studies, a relatively small number of observational units produce data used to test hypotheses for tens of thousands of genes. This is a $n \ll p$ problem with the potential of producing many incorrect results, due to random noise. To mitigate this problem, we propose a nonparametric model which considers the set of regression parameters for each gene as an i.i.d. random variable having a joint distribution with an unspecified form.

Borrowing information across genes, the model enables learning with respect to this joint distribution, providing appropriate regularization of the posterior distributions of the gene-specific parameters. To perform inference, we tap the fine-grained parallel computing power of general purpose graphics processing units (GPUs) to do Gibbs sampling. We present the results of a series of simulation studies which looked at the computational viability of our implementation along with assessments of model performance relative to competing methods. These show that our implementation is viable, can provide better overall precision for the gene-specific parameters, provides competitive accuracy in selecting genes according to their probable concordance with a hypothesis and, being fully Bayesian, provides substantially more flexibility in inference using the full posterior distribution.

Second, we present a method for joint estimation of multiple lifetime distributions based on the Generalized Limited Failure Population (GLFP) model. This 5-parameter model for lifetime data accommodates lifetime distributions with multiple failure modes:  early failures due to ``infant mortality'' and failures due to wearout. We fit the GLFP model using a hierarchical modeling approach.  Borrowing strength across populations, our method enables estimation with uncertainty of lifetime distributions even in cases where the number of model parameters is larger than the number of observed failures.  Moreover, using our Bayesian method, comparison of different product brands is straightforward because estimation of arbitrary functionals is easy using draws from the joint posterior distribution of the model parameters. Potential applications include assessment and comparison of reliability to inform purchasing decisions.