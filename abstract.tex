\section*{Abstract}
Advances in modern computing have encouraged statisticians to fit larger and larger models to larger and more complex data sets. Bayesian hierarchical models are a class of models, suitable for a wide range of applications, that offer the analyst flexibility and for which general strategies for inference have been developed. In this work we present two such models, both motivated by real applications, and develop methodologies for performing inference.

First, we present a Bayesian nonparametric hierarchical regression model for gene expression profiling data. In gene profiling studies, a relatively small number of observational units produce data used to test hypotheses for tens of thousands of genes. This is a $p \ll n$ problem with the potential of producing many incorrect results, due to random noise. To mitigate this problem, we propose a nonparametric model which considers the set of regression parameters for each gene as independ, identically distributed random variables, having a joint distribution with an unspecified form.

% Borrowing information across genes, the model enables learning with respect to this joint distribution, providing appropriate regularization of the posterior distributions of the gene-specific parameters. To perform inference, we tap the fine-grained parallel computing power of general purpose graphics processing units (GPUs) to do Gibbs sampling. We present the results of a series of simulation studies which looked at the computational viability of our implementation along with assessments of model performance relative to competing methods. These show that our implementation is viable, can provide better overall precision for the gene-specific parameters, provides competitive accuracy in selecting genes according to their probable concordance with a hypothesis and, being fully Bayesian, provides substantially more flexibility in inference using the full posterior distribution.

Second, we present a method for estimation of lifetime for populations exhibiting heterogeneity due to infant mortality. Specifically, we consider the case where multiple such populations are of interest and information for some populations is limited by censoring and truncation. We demonstrate our method on a large set of field reliability data collected on hard drives.